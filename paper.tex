\documentclass[a4paper]{article}

\usepackage[T1]{fontenc}

\author{Antoine Passe-Mieux, Théo Vers L'Est, Robin Très Petit}
\title{Bayesian inference of Markovian processes with Laplacian/Gaussian priors
using an approximate pendulum-based Kolmogorov-Smirnoff test}

\begin{document}
\maketitle

\section{Introduction}
As Aristotle exposed in 344 BC, aesthetics are unrelated to the existence of
universals (\emph{Qualia}, or \emph{hylomorphism}).
However, it is well-known form Parmenides that “You must learn all things, both the unshaken heart of persuasive truth, and the opinions of mortals in which there is no true warranty” (Frag B 1.28-30, quoted by Sextus Empiricus, \emph{Against the Mathematicians}, vii. 3; Simplicius, \emph{Commentary on the Heavens}, 557-8; Proclus, \emph{Commentary on the Timaeus I}, 345)

\section{Theory}
Bonjour

\section{Algorithms}

\section{Discussion}

\section{Acknowledgements}

\begin{itemize}
	\item Boltzmann
	\item Markov
	\item Newton
	\item Shannon
	\item Bayes
	\item Laplace
	\item Gauss
	\item Kolmogorov
	\item Smirnoff
	\item Pythagore
	\item Aristotle
	\item Occam
	\item Condorcethylomorphism
	\item Hahnemann (homeopathy)
	\item Pearl
	\item Von Neumann
	\item Turing
	\item Lagrange
	\item Dijkstra
	\item Kruskal
	\item Nash (equilibrium being his \emph{only} contribution)
	\item Bueno De Mesquita
	\item David Hillbert
	\item Daniel Bernoulli
	\item Mercer
	\item Cantor
	\item Gödel
	\item Galois
	\item Boole
	\item Yann Lecun (only pre-2000 publications)
\end{itemize}

\end{document}
